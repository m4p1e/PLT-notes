\documentclass{article}

\usepackage{ctex}
\usepackage{tikz}
\usetikzlibrary{cd}

\usepackage{amsthm}
\usepackage{amsmath}
\usepackage{amssymb}

%\usepackage{unicode-math}


\usepackage[textwidth=18cm]{geometry} % 设置页宽=18

\usepackage{blindtext}
\usepackage{bm}
\parindent=0pt
\setlength{\parindent}{2em} 
\usepackage{indentfirst}

\usepackage{listings}
%\usepackage{minted}% hightlighting

\usepackage{proof} % infer

\usepackage{xcolor}
\usepackage{titlesec}
\titleformat{\section}[block]{\color{blue}\Large\bfseries\filcenter}{}{1em}{}
\titleformat{\subsection}[hang]{\color{red}\Large\bfseries}{}{0em}{}
%\setcounter{secnumdepth}{1} %section 序号

\newtheorem{theorem}{Theorem}[section]
\newtheorem{lemma}[theorem]{Lemma}
\newtheorem{corollary}[theorem]{Corollary}
\newtheorem{proposition}[theorem]{Proposition}
\newtheorem{example}[theorem]{Example}
\newtheorem{definition}[theorem]{Definition}
\newtheorem{remark}[theorem]{Remark}
\newtheorem{exercise}{Exercise}[section]
\newtheorem{annotation}[theorem]{Annotation}

\newcommand*{\xfunc}[4]{{#2}\colon{#3}{#1}{#4}}
\newcommand*{\func}[3]{\xfunc{\to}{#1}{#2}{#3}}

\newcommand\Set[2]{\{\,#1\mid#2\,\}} %集合
\newcommand\SET[2]{\Set{#1}{\text{#2}}} %

\begin{document}
\title{Types and Programming Language}
\author{枫聆}
\maketitle
\tableofcontents

\newpage
\section{Introduction}

\begin{definition}
\rm A {\color{red} type system} is a tractable syntactic method for proving the absence of certain program behaviors by classlying phrases according to the kinds of value they compute.
\end{definition}

{\color{blue} type system是一种用于证明某些确定的程序行为不会发生的方法,它怎么做呢?通过它们计算出值的类型来分类, 有点抽象... 我想知道the kinds of value they compute是什么? 如何分类? 分类之后接下来该怎么做?}

\begin{annotation}
\rm Being static, type systems are necessarily also {\color{red}conservative}: they can categorically prove the absence of some bad program behaviors, but they cant prove their presence.
\end{annotation}

\lstset{language=C,
    basicstyle=\ttfamily,
    keywordstyle=\bfseries,
    showstringspaces=false,
    morekeywords={include, printf}
}
\lstset{numbers=left, numberstyle=\tiny, stepnumber=2, numbersep=5pt}

\begin{example}
\rm
\begin{lstlisting}
if <complex test> then 5 else <type error>
\end{lstlisting}
{\color{blue} 上面这个annotation在说type system只能证明它看到的一些bad program behavior不会出现,但是它们可能会reject掉一些runtime time阶段运行良好的程序,例如在runtime阶段上面的else可能永远都不会进. 即type system无法证明它是否真的存在}.
\end{example}

\newpage
\section{Untyped Systems}

\subsection{Syntax}

\begin{definition}
\rm The set of terms is the smallest set $\mathcal{T}$ such that 
\begin{enumerate}
	\item $\{\text{true},\text{false},0\} \subseteq \mathcal{T}$;
	\item if $t_1 \in \mathcal{T}$, then $\{\text{succ}\ t_1, \text{pred}\ t_1, \text{iszero}\ t_1\} \subseteq \mathcal{T}$;
	\item if $t_1 \in \mathcal{T}, t_2 \in \mathcal{T}, t_3 \in \mathcal{T}$, then {\lstinline[mathescape]!if $t_1$ then $t_2$ else $t_3$!} $\in \mathcal{T}$.
\end{enumerate}
\end{definition}

\begin{definition}
\rm The set of terms is defined by the following rules:
$$
\begin{array}{ccc}
\text{true} \in \mathcal{T}  & \text{false} \in \mathcal{T} & 0 \in \mathcal{T} \\
\infer{\text{succ} t_1 \in \mathcal{T}}{t_1 \mathcal{T}} & \infer{\text{succ} t_1 \in \mathcal{T}}{t_1 \mathcal{T}} & \infer{\text{succ} t_1 \in \mathcal{T}}{t_1 \mathcal{T}} \\
& \infer{\text{\lstinline[mathescape]!if $t_1$ then $t_2$ else $t_3$!}}{ t_1 \in \mathcal{T} & t_2 \in \mathcal{T} & t_3 \in \mathcal{T}} &\\ 
\end{array}
$$
\end{definition}

\begin{definition}
\rm For each natural number $i$, define a S(X) as follow:
$$
\begin{aligned}
S_0(X)  &= X \\
S_1(X)  &= \Set{\text{succ}\ t,\text{prev}\ t,\text{iszero}\ t }{t \in X} \cup \Set{\text{\lstinline[mathescape]!if $t_1$ then $t_2$ else $t_3$!}}{t_1,t_2,t_3 \in X} \\
&\vdots \\
S_{i+1}(X) &= S(S_i(X)).
\end{aligned}
$$
\end{definition}

\begin{proposition}
\rm $\mathcal{T} = \bigcup_{i = 0}^{\omega} S_i(\{\text{true},\text{false},0\})$.
\end{proposition}

\begin{proof}

\end{proof}


\end{document}