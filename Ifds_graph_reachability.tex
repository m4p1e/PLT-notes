\documentclass{article}

\usepackage{ctex}
\usepackage{tikz}
\usetikzlibrary{cd}

\usepackage{amsthm}
\usepackage{amsmath}
\usepackage{amssymb}

\usepackage[linesnumbered,ruled,vlined]{algorithm2e}

%\usepackage{unicode-math}

\usepackage{hyperref} %url
\hypersetup{
    colorlinks=true,
    linkcolor=blue,
    filecolor=magenta,      
    urlcolor=cyan,
    pdftitle={Overleaf Example},
    pdfpagemode=FullScreen,
    }


\usepackage[textwidth=18cm]{geometry} % 设置页宽=18

\usepackage{blindtext}
\usepackage{bm}
\parindent=0pt
\setlength{\parindent}{2em} 
\usepackage{indentfirst}


\usepackage{xcolor}
\usepackage{titlesec}
\titleformat{\section}[block]{\color{blue}\Large\bfseries\filcenter}{}{1em}{}
\titleformat{\subsection}[hang]{\color{red}\Large\bfseries}{}{0em}{}
%\setcounter{secnumdepth}{1} %section 序号

\newtheorem{theorem}{Theorem}[section]
\newtheorem{lemma}[theorem]{Lemma}
\newtheorem{corollary}[theorem]{Corollary}
\newtheorem{proposition}[theorem]{Proposition}
\newtheorem{example}[theorem]{Example}
\newtheorem{definition}[theorem]{Definition}
\newtheorem{remark}[theorem]{Remark}
\newtheorem{exercise}{Exercise}[section]
\newtheorem{annotation}[theorem]{Annotation}

\newcommand*{\xfunc}[4]{{#2}\colon{#3}{#1}{#4}}
\newcommand*{\func}[3]{\xfunc{\to}{#1}{#2}{#3}}

\newcommand\Set[2]{\{\,#1\mid#2\,\}} %集合
\newcommand\SET[2]{\Set{#1}{\text{#2}}} %

\newcommand{\redt}[1]{\textcolor{red}{#1}}
\newcommand{\bluet}[1]{\textcolor{blue}{#1}}

\begin{document}
\title{IFDS: Dataflow Analysis via Graph Reachability}
\author{枫聆}
\maketitle
\tableofcontents

\newpage
\section{Definitions}


\begin{annotation}
\rm 在数据流分析中的“精确”一词,实际上等价于“meet over all vaild path”. 
\begin{itemize}
	\item 在过程内分析(intraprocedural)中,一条“vaild path”就是指从某个procedural的CFG上从entry node到特定的点这样一条路径. 
	\item 在过程间分析(interprocedural)中,一条“vaild path”就是指当从main function开始,且某个procedural结束之后返回调用它的procedural,直到某个特定程序点的这样一条路径.  
\end{itemize}
上述东西没有什么新意,但是让各种名词形式化有利于表达. 
\end{annotation}

\begin{definition}
\rm 数据流分析中的可能会出现所有不同的数据值组成的集合$D$(underlying set)称为dataflow facts. 对于可能分析得到的结果是dataflow facts的一个子集,通常我们把所用可能得到的结果记为$2^D$. 
\end{definition}

\begin{definition}
\rm 数据流的值可以表示成位向量(bit-vectors),其中每个bit可以表示一个具体的dataflow fact, 且可以每个传递函数可以用相应的位运算来表示,这样的一类数据流分析问题我们称之为\redt{locally separable problems}. i.e. reaching-definitions, available expressions, live variables. 
\end{definition}

\begin{annotation}
\rm \redt{怎么理解"separable"?} separable对应的是位运算过程不同位bit是不会相互相互影响的,也就是两个不同dataflow fact是不会相互依赖的. 例如在reaching-definitions中两个不同变量的定义的作用域是不会相互影响的.   
\end{annotation}

\begin{definition}
\rm 
\end{definition}

\end{document}